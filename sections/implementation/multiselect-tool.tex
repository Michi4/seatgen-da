A very important feature was enabling the user to select multiple seats at once. This was especially crucial because a venue can have many seats, and selecting them one by one would be very time-consuming. A common operation for the user is also moving entire sectors. To tackle this challenge, the decision was made to develop the multiselect tool. The multiselect tool draws a rectangle when selected and dragged on the map, selecting everything inside this rectangle. Leaflet already provides a feature that uses a rectangle, which works on the user's drag interaction. This feature is called BoxZoom. The BoxZoom feature is a built-in capability of Leaflet that allows the user to draw a rectangle on the map and zoom into the area of the rectangle. This feature served as a good starting point for the multiselect tool because it already provides rectangle drawing and drag interaction. The BoxZoom feature was extended to develop the multiselect tool. The multiselect tool is a subclass of the BoxZoom feature and overrides the functions responsible for zooming. Instead of zooming, the multiselect tool selects all the seats inside the rectangle. The multiselect tool is shown in listing \ref{lst:multiselect-tool}. The finished functionality for selecting seats is shown in figure \ref{fig:multiselect-tool}.
\begin{figure}
    \centering
    \includegraphics[scale=0.22]{pics/multiselect.png}
    \caption{Multiselect Tool}
    \label{fig:multiselect-tool}
\end{figure}

\begin{lstlisting}[language=Typescript, caption={Multiselect Tool},label={lst:multiselect-tool}]
const MapBoxSelect = (props) => {
    const propsRef = useRef(props);

    useEffect(() => {
        propsRef.current = props;
    }, [props]);

    let _startPoint;
    let _currFunction;

    function checkIfNothingSelected() {
        const currentProps = propsRef.current;
        return (currentProps.currentSelectedTool === undefined || currentProps.currentSelectedTool === null || !(currentProps.currentSelectedTool.id in currentProps.handleDrag));
    }

    L.Map.Multiselect = L.Map.BoxZoom.extend({
        _onMouseDown: function (e) {
            if (checkIfNothingSelected()) {
                return false;
            }
            _currFunction = propsRef.current.handleDrag[propsRef.current.currentSelectedTool.id]
            L.DomUtil.disableTextSelection();
            propsRef.current.mapRef.current?.dragging.disable()

            _startPoint = this._map.mouseEventToLayerPoint(e);

            this._box = L.DomUtil.create('div', 'leaflet-zoom-box', this._pane);
            L.DomUtil.setPosition(this._box, this._startLayerPoint);

            this._container.style.cursor = propsRef.current.currentSelectedTool;

            L.DomEvent
                .on(document, 'mousemove', this._onMouseMove, this)
                .on(document, 'mouseup', this._onMouseUp, this)
                .on(document, 'keydown', this._onKeyDown, this)
                .preventDefault(e);

            this._map.fire('boxzoomstart');
        },

        _onMouseMove: function (e) {
            if (checkIfNothingSelected()) {
                return false;
            }
            var startPoint = _startPoint,
                box = this._box,

                layerPoint = this._map.mouseEventToLayerPoint(e),
                offset = layerPoint.subtract(startPoint),

                newPos = new L.Point(
                    Math.min(layerPoint.x, startPoint.x),
                    Math.min(layerPoint.y, startPoint.y));

            L.DomUtil.setPosition(box, newPos);

            box.style.width = (Math.max(0, Math.abs(offset.x) - 4)) + 'px';
            box.style.height = (Math.max(0, Math.abs(offset.y) - 4)) + 'px';
        },

        _onMouseUp: function (e) {
            if (checkIfNothingSelected()) {
                return false;
            }

            this._finish();
            const map = this._map,
                layerPoint = map.mouseEventToLayerPoint(e);
            const bounds = new L.LatLngBounds(
                map.layerPointToLatLng(layerPoint),
                map.layerPointToLatLng(_startPoint)
            )

            if (_currFunction != null) {
                _currFunction(bounds);
            }
            _currFunction = null
        },

        _finish: function () {
            propsRef.current.mapRef.current?.dragging.enable()
            if (Array.from(this._pane.children).includes(this._box)) {
                this._pane.removeChild(this._box);
            }
            this._container.style.cursor = '';

            L.DomUtil.enableTextSelection();

            L.DomEvent
                .off(document, 'mousemove', this._onMouseMove)
                .off(document, 'mouseup', this._onMouseUp)
                .off(document, 'keydown', this._onKeyDown);
        },

    })

    L.Map.mergeOptions({boxSelect: true});
    L.Map.addInitHook('addHandler', 'boxSelect', L.Map.Multiselect);

    L.Map.mergeOptions({boxZoom: false});
}

export default MapBoxSelect
\end{lstlisting}

This code uses parts of the original code concepts, and adapts it for the selection of seats. The original source can be found in Leaflet's source code.

This box is utilized not only for the multiselect tool, but also for the grid tool which is explained in more detail in section \ref{sec:grid-tool}. The modification of the ZoomBox only needs to be done once, because they can be reused by both tools. The modified functionality just disables the zoom of the original feature, and accepts a function that is called onMouseUp with the boundaries of the drawn rectangle as parameters. When the rectangle select is needed, it can be dynamically loaded into the map component.

Except for this tool, another way of selecting multiple seats at once was implemented, because of usability reasons and user expectations. In editors ranging from Photoshop, Gimp, to File Explorers, holding the \texttt{strg} or \texttt{cmd} key while clicking on an object allows the user to select multiple objects. This was also implemented in SeatGen. The user can hold this key and click on a seat to select more than one seat.