To maintain a modular and scalable UI, SeatGen’s frontend follows a structured component-based architecture using React. Each UI section, including the map, toolbar, menus, and detail editor, is encapsulated in independent components. These components communicate through React’s state management system and the map context to ensure that UI updates remain efficient and consistent.

\begin{figure}[H]
    \begin{center}
        \includegraphics[scale=0.3]{pics/frontend_architecture.png}
    \end{center}
    \caption{Frontend Component Architecture (Rough Overview)}
    \label{fig:frontend-architecture}
\end{figure}

The diagram in Figure \ref{fig:frontend-architecture} illustrates the core interactive components of the SeatGen frontend. However, it does not represent all UI elements, such as the menu, landing page, modals, and settings, which are also essential for the overall user experience.

Beyond the primary seating map, detail editor, and tool system, SeatGen's frontend also includes:

\begin{itemize}
    \item \textbf{Landing Page}: The first screen that users see upon opening SeatGen. It provides an introduction to the tool and a project overview for first-time users.
    
    \item \textbf{Home Menu}: This serves as the main project selection interface. Users can choose between loading an already saved seating plan or creating a new one. The UI provides a minimal and intuitive interface for quick access to projects.
    
    \item \textbf{Menu and Navigation}: The global navigation system connects different views within SeatGen. It allows users to switch between the seating map editor, settings, and export functionalities. The navigation is designed to remain persistent throughout the application to provide seamless transitions between different tasks.
    
    \item \textbf{Settings Panel}: This panel allows users to configure global preferences that influence the overall seating arrangement experience. Options include:
    \begin{itemize}
        \item \textbf{Default Seat Categories}: Users can predefine categories to streamline seat assignments.
        \item \textbf{Theme Selection}: Dark or light mode, based on user preference.
        \item \textbf{Seat Map Scaling}: Allows fine-tuned adjustments of grid seat density.
    \end{itemize}
    
    \item \textbf{Modals and Popups}: The frontend includes various popups and confirmation dialogs:
    \begin{itemize}
        \item \textbf{Modals}: Used for confirmations, warnings, and additional actions.
        \item \textbf{Leave Page Popup}: Warns users about unsaved changes before leaving.
    \end{itemize}
\end{itemize}

Although not included in the diagram, these elements play a significant role in efficiently navigating and managing seat plans, further enhancing the user experience.


\subsection{Map}
\setauthor{Michael Ruep}


The \textbf{MapComponent.tsx} is one of the most critical components in SeatGen, responsible for rendering and managing the stadium seating map. It integrates with \textbf{Leaflet.js} to provide interactive seat visualization, selection, and manipulation.

\textbf{Key Responsibilities:}
\begin{itemize}
    \item Initializes and manages the \textbf{Leaflet map instance}.
    \item Loads seat and standing area data dynamically from the backend.
    \item Renders seats and standing areas as \textbf{interactive markers and polygons}.
    \item Handles user interactions such as \textbf{clicking, selecting, and dragging seats}.
    \item Implements \textbf{state synchronization} with the global context (\textbf{MapContext.tsx}).
    \item Provides support for \textbf{tool-based seat editing} (e.g., adding, deleting, moving).
\end{itemize}

\textbf{Component Structure:}
\begin{itemize}
    \item \textbf{State Management:} Uses \texttt{useState}, \texttt{useEffect}, and \texttt{useContext} to manage map data.
    \item \textbf{Leaflet Integration:} Utilizes \texttt{MapContainer} from \texttt{react-leaflet} for seamless map rendering.
    \item \textbf{Performance Enhancements:} Implements \texttt{useCallback} and \texttt{useRef} to optimize re-renders.
\end{itemize}

\subsubsection{Map Initialization and Leaflet Integration}

The \textbf{MapComponent.tsx} initializes the Leaflet map using the react-leaflet \texttt{MapContainer} component.

\begin{lstlisting}[language=TypeScript, caption=Initializing Leaflet Map in React, label=lst:react-leaflet]
const MapComponent: FC<MapProps> = ({ editable: initialEditable }) => {
    const context = useMapContext(); // Access global state (MapContext.tsx)
    const { bucketName, mapName } = useParams(); // Retrieve map identifiers from URL params

    const [editable, setEditable] = useState(initialEditable ?? true);
    const mapRef = useRef<L.Map>(null);

    return (
        <MapContainer
            crs={L.CRS.Simple} // Uses a flat, pixel-based coordinate system
            className="leaflet-container h-full w-full"
            ref={mapRef}
            center={[context.mapInfo.tileSize / (-2), context.mapInfo.tileSize / (2)]}
            zoom={context.mapInfo.defaultZoom}
            maxZoom={context.mapInfo.maxZoom}
            minZoom={context.mapInfo.minZoom}
            scrollWheelZoom={true}
            zoomControl={false}
            doubleClickZoom={false}
            preferCanvas={true}
            dragging={true}
            tap={false}
            renderer={L.canvas()} // Use Canvas for better performance
        >
            <TileLayer tms={true} url={`${context.mapInfo.mapDto.baseUrl}/{z}/{x}/{y}.png`} />
        </MapContainer>
    );
};
\end{lstlisting}

\textbf{Custom CRS (Coordinate Reference System):}
\begin{itemize}
    \item Uses \textbf{L.CRS.Simple}, a \textbf{2D pixel-based coordinate system}.
    \item Unlike traditional geographic maps, SeatGen does not need a curved map.
    \item All coordinates can be converted to a Cartesian X/Y grid.
\end{itemize}

\textbf{Custom Rendering Engine:}
\begin{itemize}
    \item Uses \textbf{L.canvas()} instead of SVG for performance.
    \item Enables handling of thousands of seat markers.
    \item Reduces DOM load by rendering elements in a drawing surface.
\end{itemize}

\textbf{Dynamically Loading Map Tiles:}
\begin{itemize}
    \item The tile URL is dynamically constructed based on \texttt{context.mapInfo}.
    \item Tiles are loaded asynchronously to improve map load times.
\end{itemize}


\subsubsection{Fetching Seat and Standing Area Data}
\setauthor{Michael Ruep}

The MapComponent.tsx fetches seat and standing area data from the backend when the component mounts. This is done using the api client in the useEffect hook.

\begin{lstlisting}[language=TypeScript, caption=Fetching Seat and Category Data, label=lst:fetch-seats]
useEffect(() => {
    if (bucketName && mapName) {
        // Fetch basic stadium map information
        seatgenApiClient.api.info(bucketName, mapName)
            .then(response => context.setMapInfo(response.data))
            .catch(error => console.error('Error fetching map info:', error));

        // Fetch seat categories before retrieving individual seats
        seatgenApiClient.api.getAllCategories().then((r) => {
            if (r.ok) {
                context.setCategories(r.data);
                seatgenApiClient.api.getSeatsByMap(bucketName, mapName).then(response => {
                    if (r.ok) {
                        context.setSeats(response.data.map((s) => ({
                            id: s.seatId!,
                            position: { lat: s.xcoord!, lng: s.ycoord! },
                            category: r.data.find(it => it.id === s.categoryId) ?? null
                        })));
                    }
                });
            }
        });

        // Fetch standing areas
        seatgenApiClient.api.getSectorByMap(bucketName, mapName).then((r) => {
            if (r.ok) {
                context.setStandingAreas(r.data.map((data) => ({
                    id: data.id!,
                    name: data.name!,
                    capacity: data.capacity,
                    coordinates: data.coordinates?.map((c) => new L.LatLng(c.x!, c.y!)) ?? [],
                    selected: false
                })));
            }
        });

    } else {
        console.error("BucketName or MapName not set");
    }
}, [bucketName, mapName]);
\end{lstlisting}

\textbf{Breakdown of Logic:}
\begin{itemize}
    \item Fetches stadium metadata (mapInfo) and sets it globally.
    \item Retrieves seat categories.
    \item Retrieves seat positions from the backend and maps them into React state.
    \item Ensures data consistency by linking seats to their corresponding categories.
\end{itemize}

\textbf{Example of Mapped Seat Object:}
\begin{lstlisting}[language=TypeScript, caption=Seat Object in State, label=lst:seat-object]
{
    id: 1234,
    position: { lat: 48.3069, lng: 14.2858 }, // Example coordinates
    category: { id: 2, name: "VIP", color: "#FFD700" } // Associated category
}
\end{lstlisting}

\subsubsection{Fetching Standing Area / Sector Data}
In addition to seats, the component also retrieves standing areas, which are handled separately.

\begin{lstlisting}[language=TypeScript, caption=Fetching Standing Areas, label=lst:fetch-standingareas]
seatgenApiClient.api.getSectorByMap(bucketName, mapName).then((r) => {
    if (r.ok) {
        context.setStandingAreas(r.data.map((data) => ({
            id: data.id!,
            name: data.name!,
            capacity: data.capacity,
            coordinates: data.coordinates?.map((c) => new L.LatLng(c.x!, c.y!)) ?? [],
            selected: false
        })));
    }
});
\end{lstlisting}

\textbf{Explanation:}
\begin{itemize}
    \item The API returns a list of sector polygons.
    \item Each area consists of a unique ID, name, capacity, and a list of coordinates.
    \item Coordinates are transformed into Leaflet’s LatLng format to be rendered as a polygon.
\end{itemize}

\textbf{Example of a Standing Area Object in State:}
\begin{lstlisting}[language=TypeScript, caption=Standing Area Object in State, label=lst:standingarea-object]
{
    id: 5678,
    name: "Sektor 1",
    capacity: 500,
    coordinates: [
        { lat: 48.3069, lng: 14.2858 },
        { lat: 48.3075, lng: 14.2862 },
        { lat: 48.3080, lng: 14.2856 }
    ],
    selected: false
}
\end{lstlisting}

\subsubsection{State Management and Performance}
\begin{itemize}
    \item \texttt{useEffect} Dependency Array:
    \begin{itemize}
        \item Ensures the API calls only run when \texttt{bucketName} or \texttt{mapName} change.
        \item Prevents unnecessary re-fetching on every render.
    \end{itemize}
    
    \item Efficient State Updates:
    \begin{itemize}
        \item Avoids unnecessary re-renders by batching state updates for seats and standing areas.
        \item No prop-drilling by storing globaly needed fetched data in the \texttt{MapContext}.
    \end{itemize}

    \item Error Handling:
    \begin{itemize}
        \item If any API call fails, an error is logged, and the operation is skipped.
        \item Ensures that failures in one request do not crash the entire component.
    \end{itemize}
\end{itemize}

\subsubsection{Rendering Seats and Handling Selection}
\setauthor{Michael Ruep}

Each seat in the stadium is rendered as a Leaflet marker, allowing users to interact with them dynamically. The selection mechanism is designed to provide an intuitive experience while supporting multi-selection for bulk operations.

\textbf{Rendering Seat Markers:}
\begin{lstlisting}[language=TypeScript, caption=Rendering and Selecting Seats, label=lst:seat-rendering]
const handleSeatClick = useCallback((id: number, event: L.LeafletMouseEvent) => {
    const isCtrlPressed = event.originalEvent?.ctrlKey;

    context.setSeats(prevSeats => prevSeats.map(seat => {
        if (seat.id === id) {
            const selected = isCtrlPressed ? !seat.selected : true;
            return { ...seat, selected };
        }
        return isCtrlPressed ? seat : { ...seat, selected: false };
    }));

    setOpenSideBar(true);
}, [context]);
\end{lstlisting}

\textbf{How This Works:}
\begin{itemize}
    \item Clicking a seat toggles its selected state.
    \item Holding \textbf{Ctrl} allows multi-selection, useful for bulk actions.
    \item Clicking a seat opens the \textbf{DetailEditor sidebar} for further modifications.
\end{itemize}


\subsection{Detail Editor}
\setauthor{Michael Ruep}
The detail editor component provides an interface for modifying selected seats, managing seat groups, categories, and configuring standing areas. It plays a huge role in SeatGen’s user interaction system by allowing users to efficiently modify stadium layouts in a structured and intuitive manner.

\begin{wrapfigure}{r}{0.4\textwidth}
    \begin{center}
    \includegraphics[width=0.4\textwidth]{pics/DetailEditorSeat.png}
    \caption{Seat Editing Panel in Detail Editor}
    \label{fig:detail-editor-seat}
    \end{center}
\end{wrapfigure}

The core functionalities of the detail editor include:
\begin{itemize}
    \item \textbf{Editing Individual Seats:} Users can update seat tooltips, positions, and categories.
    \item \textbf{Managing Seat Groups:} Enables users to create, merge, and delete groups of seats.
    \item \textbf{Standing Area Editing:} Supports renaming and capacity adjustments for standing areas.
    \item \textbf{Category Assignment:} Allows users to assign pricing tiers and colors to seats.
\end{itemize}

\subsection{Component Structure}
The detail editor has the following core sections:

\begin{itemize}
    \item \textbf{Seat Editing Panel:} Displays detailed information for selected seats and allows modifications.
    \item \textbf{Group Management Panel:} Handles seat grouping and bulk operations.
    \item \textbf{Standing Area Editing Panel:} Enables editing of standing areas, including their name and capacity.
    \item \textbf{Deletion and Bulk Actions:} Supports removing selected seats, groups, or selected standing areas.
\end{itemize}

Figure \ref{fig:detail-editor-seat} illustrates the seat editing panel, where users can adjust tooltips, assign categories, and delete seats.

\subsection{Managing Seat Attributes}
When a seat is selected, the editor provides multiple options for modification.

\subsubsection{Updating Tooltip Text}
Users can edit tooltip descriptions for seats, making it easier to add position information or other special notes.

\begin{lstlisting}[language=TypeScript, caption=Updating Tooltip for Selected Seats, label=lst:update-tooltip]
<input
    type="text"
    value={props.currentTooltip}
    onChange={props.handleExternalTooltipChange}
    placeholder="Edit Tooltip Text"
/>
\end{lstlisting}

This behavior works as follows:
\begin{itemize}
    \item The text input dynamically updates the tooltip for all selected seats by calling the \texttt{handleExternalTooltipChange} function.
    \item Changes are stored in the global state.
    \item Provides immediate visual feedback to the user.
\end{itemize}

\subsubsection{Adjusting Seat Position via UI}
Instead of manually dragging seats on the map, users can fine-tune their positions numerically. Also, adjusting the steps of one increment when using the arrow keys to position the seats perfectly is possible.

\begin{lstlisting}[language=TypeScript, caption=Updating Seat Coordinates, label=lst:update-seat-position]
const updateSeatPosition = (x: number, y: number) => {
    const latLng = props.mapRef.current!.layerPointToLatLng(new L.Point(x, y));
    props.updateSelectedSeatsPosition(selectedSeats[0].id, { lat: latLng.lat, lng: latLng.lng }, true);
};
\end{lstlisting}

The following sequence describes how this interaction is handled internally:
\begin{itemize}
    \item Converts user-inputted X/Y values into map coordinates.
    \item Updates seat position dynamically.
    \item This optimizes movement for both manual entry and real-time adjustments.
\end{itemize}

\subsection{Group Management and Multi-Selection}
Grouping seats allows users to efficiently manage large stadium sections.
\subsubsection{Seat Groups, Multi-Selection, and Deletion}

SeatGen supports the concept of seat groups, allowing users to efficiently manipulate multiple seats at once. Grouping seats enables bulk operations such as movement, category assignment, and section-wide modifications, making it particularly useful for managing large stadium layouts.

The following scenarios demonstrate practical applications for seat grouping:
\begin{itemize}
    \item \textbf{Bulk Editing:} Modify multiple seat attributes simultaneously.
    \item \textbf{Efficient Repositioning:} Move multiple seats while maintaining relative positioning.
    \item \textbf{Category Assignment:} Assign pricing tiers and access restrictions to an entire section.
    \item \textbf{Simplified Deletion:} Remove entire seat groups without manually selecting each seat.
\end{itemize}

The Listing \ref{lst:create-seat-group} demonstrates how a new group is created from currently selected seats:
\begin{lstlisting}[language=TypeScript, caption=Creating Seat Groups, label=lst:create-seat-group]
const createGroup = useCallback(() => {
    context.doAction(new CreateGroupAction(setSeatGroups, selectedSeats));
}, [selectedSeats, context]);
\end{lstlisting}

This mechanism works through the following sequence:
\begin{itemize}
    \item The function retrieves all currently selected seats.
    \item The CreateGroupAction is executed, registering the selected seats as a group.
    \item The group id is assigned, and the group is stored in the global state.
\end{itemize}

To remove a previously created seat group, the logic shown in Listing \ref{lst:delete-seat-group} is used:
\begin{lstlisting}[language=TypeScript, caption=Deleting Seat Groups, label=lst:delete-seat-group]
const deleteGroup = useCallback((groupId: number) => {
    context.doAction(new DeleteGroupAction(setSeatGroups, groupId));
}, [context]);
\end{lstlisting}

The group deletion logic is carried out as follows:
\begin{itemize}
    \item The DeleteGroupAction removes that group using the groupId.
    \item The global context state updates accordingly.
    \item The operation is reversible via the undo stack.
\end{itemize}

Furthermore, users can merge existing groups or split them dynamically, enabling flexible seat management. This allows for unlimited subgrouping, making the editor more intuitive and efficient.

\subsubsection{Seat Categories}
\setauthor{Michael Ruep}

Seat categories allow users to classify seating arrangements based on (pricing) tiers, accessibility, and special designations such as VIP areas or restricted sections. In SeatGen, every seat can be assigned a category that determines its visual representation and ticketing attributes.

\begin{figure}[H]
    \centering
    \includegraphics[width=0.5\textwidth]{pics/DetailEditorCategory.png}
    \caption{Category Selection in Detail Editor}
    \label{fig:detail-editor-category}
\end{figure}

The following list outlines the core features supported by seat categories:
\begin{itemize}
    \item \textbf{Color-Coding:} Each category is assigned a color for clear visualization.
    \item \textbf{Pricing Information:} Categories define pricing tiers, ensuring correct ticket pricing.
    \item \textbf{Flexible Assignments:} Seats can be reassigned to different categories as needed.
    \item \textbf{Bulk Category Updates:} Multiple seats can be assigned a category simultaneously.
\end{itemize}

Each seat category is stored as an object that holds classification data:

\begin{lstlisting}[language=TypeScript, caption=Seat Category Data Model, label=lst:seat-category-model]
interface Category {
    id: number;
    name: string; // Example: "VIP", "General Admission"
    color: string; // Hex code for UI representation
    price: number; // Ticket price associated with this category
}
\end{lstlisting}

When a user selects a seat, they can assign or change its category using the detail editor.

\begin{lstlisting}[language=TypeScript, caption=Update Category Action Call, label=lst:assign-category]
onChange={(newVal) => {
    context.doAction(new UpdateCategoryAction(selectedSeats, newVal ? newVal.value : null))
    setCurrentSelectedCategory(newVal ? newVal.value : null);
}}
\end{lstlisting}

\begin{lstlisting}[language=TypeScript, caption=Update Category Action, label=lst:assign-category]
execute = () => {
    this._oldCategories = this._selectedSeats.map(seat => seat.category)
    this._selectedSeats.forEach((seat) => {
        seat.category = this._newCategory
    });
}
\end{lstlisting}

This assignment works by executing the following steps:
\begin{itemize}
    \item The function in \texttt{UpdateCategoryAction} loops through all selected seats.
    \item The new category is assigned based on the provided \texttt{categoryId}.
    \item The UI updates instantly, applying the new color and classification.
\end{itemize}

Users can also manage available categories directly within the interface:
\begin{itemize}
    \item Creating new categories with custom colors and pricing.
    \item Editing existing categories, updating names, prices, or colors.
    \item Deleting unused categories.
\end{itemize}

\begin{lstlisting}[language=TypeScript, caption=Managing Categories, label=lst:manage-categories]
onCreateOption={(inputValue, inputColor) =>
    seatgenApiClient.api.createCategory({name: inputValue, color: inputColor}).then((r) => {
        if (r.ok) {
            enqueueSnackbar("Successfully created category", {variant: 'success'})
            selectedSeats.forEach((seat) => {
                seat.category = r.data;
            });
            context.setCategories(prevCategories => [...prevCategories, r.data]);
            context.categoryChecksums.current.set(r.data.id!, objectHash(r.data))
            setCurrentSelectedCategory(r.data)
        } else {
            console.error('Error creating category:', r.statusText);
            enqueueSnackbar('Error creating category', {variant: 'error'});
        }
    })
}
\end{lstlisting}

\begin{lstlisting}[language=TypeScript, caption=Managing Category Color, label=lst:manage-categories]
<input 
    type="color"
    value={currentSelectedCategory.color}
    onChange={(newColor) => {
        selectedSeats[0].category!.color = newColor.target.value
        setCurrentSelectedCategory({...currentSelectedCategory, color: newColor.target.value})
    }));
}}/>
\end{lstlisting}

Seat categories are visually represented by color-coded markers in the map component. Each seat marker dynamically updates based on its assigned category.

\begin{lstlisting}[language=TypeScript, caption=Rendering Seat Markers with Categories, label=lst:render-seat-category]
const renderedSeats = seats.map(seat => (
    <SeatMarker 
        key={seat.id} 
        seat={seat} 
        color={seat.category?.color || "gray"} 
        onClick={() => handleSeatClick(seat.id)}
    />
));
\end{lstlisting}

This visualization logic functions as follows:
\begin{itemize}
    \item Each seat marker inherits the category's color.
    \item Unassigned seats default to a neutral color (gray).
    \item Selecting a seat allows users to change its category.
\end{itemize}

In summary, categories play a crucial role in SeatGen, providing a structured way to manage ticket pricing and seat classification. By integrating category assignment with group selection and bulk operations, users can efficiently update stadium layouts with minimal effort.

\subsection{Standing Area Editing}
\setauthor{Michael Ruep}
Standing areas in SeatGen are defined as polygonal sectors rather than individual seats. Unlike seats, which are represented as distinct markers, standing areas are implemented using a defined boundary of polygons. Each standing area has a name, and a maximum capacity.

The following features define the functionality of standing areas:
\begin{itemize}
    \item \textbf{Custom Polygonal Boundaries:} Users can define standing areas by selecting points on the map.
    \item \textbf{Capacity Control:} Each area has a maximum capacity limit.
    \item \textbf{Real-Time Editing:} Users can rename standing areas and adjust their capacities dynamically.
    \item \textbf{Deletion and Reconfiguration:} Existing standing areas can be removed or modified at any time.
\end{itemize}

Each standing area is stored as an object in the application state.

\begin{lstlisting}[language=TypeScript, caption=Standing Area Data Model, label=lst:standingarea-model]
interface StandingArea {
    id: number;
    name: string; // Display name of the standing area
    capacity: number; // Maximum allowed attendees
    coordinates: L.LatLng[]; // List of boundary points defining the area
    selected: boolean; // Boolean flag indicating selection state
}
\end{lstlisting}

Standing areas are created by defining polygonal boundary coordinates on the map. Once an area is selected, it becomes editable in the detail editor.

Users can rename standing areas directly in the detail editor panel.

\begin{lstlisting}[language=TypeScript, caption=Renaming Standing Areas, label=lst:rename-standingarea]
const handleStandingNameChange = (name: string) => {
    context.setStandingAreas(prev => prev.map(area =>
        context.selectedStandingAreaIds.includes(area.id)
            ? { ...area, name: name }
            : area
    ));
};
\end{lstlisting}

This renaming functionality shown in Listing \ref{lst:rename-standingarea} works as follows:
\begin{itemize}
    \item The function updates the name of all selected standing areas.
    \item Changes are reflected instantly in the UI.
    \item The new name is stored persistently for future sessions.
\end{itemize}
To control attendee limits, each standing area has an adjustable capacity.

\begin{lstlisting}[language=TypeScript, caption=Updating Standing Area Capacity, label=lst:update-standingarea-capacity]
const handleStandingCapacityChange = (capacity: string) => {
    context.setStandingAreas(prev => prev.map(area =>
        context.selectedStandingAreaIds.includes(area.id)
            ? { ...area, capacity: Number(capacity) }
            : area
    ));
};
\end{lstlisting}

This capacity adjustment functionality shown in Listing \ref{lst:update-standingarea-capacity} works as follows:
\begin{itemize}
    \item The function modifies the capacity of all selected standing areas.
    \item Input validation ensures that only numeric values are accepted.
    \item The UI dynamically updates, reflecting the new ticketing constraints.
\end{itemize}

Standing areas can also be removed using \texttt{DeleteStandingAreaAction}, which supports undo operations for user convenience.

\begin{lstlisting}[language=TypeScript, caption=Deleting Standing Areas, label=lst:delete-standingarea]
context.doAction(new DeleteStandingAreaAction(selectedStandingAreas, context))
\end{lstlisting}

This deletion mechanism shown in Listing \ref{lst:delete-standingarea} works as follows:
\begin{itemize}
    \item The action removes the selected standing areas from the context.
    \item Changes are recorded in the action history, enabling undo functionality.
    \item The user interface is updated immediately to reflect the deletion.
\end{itemize}

\begin{figure}[H]
    \centering
    \includegraphics[width=0.4\textwidth]{pics/DetailEditorStandingArea.png}
    \caption{Editing Standing Areas in Detail Editor}
    \label{fig:detail-editor-standingarea}
\end{figure}

Figure \ref{fig:detail-editor-standingarea} illustrates the interface for editing standing areas, including renaming, capacity adjustment, and deletion.

Standing areas in SeatGen offer a structured approach to handling non-seated sections of a stadium. By allowing dynamic capacity control and easy renaming, the system ensures that standing areas remain flexible and adaptable. Users can efficiently create, edit, and remove standing areas based on event needs, making stadium layouts highly customizable.

Beyond general standing areas, this feature can also be used to designate specific sections for specialized needs. For example, stadiums may allocate certain sectors for wheelchair-accessible areas or priority seating for individuals with mobility impairments. This flexibility ensures that accessibility requirements can be met while maintaining a clear and organized seating plan.

\subsubsection{Summary}
The detail editor is a key component within SeatGen, offering intuitive tools for modifying stadium layouts. It enables:
\begin{itemize}
    \item \textbf{Seat Editing:} Real-time adjustments to tooltips, positions, and assignments.
    \item \textbf{Group Management:} Merging, splitting, and deleting seat groups efficiently.
    \item \textbf{Category Assignment:} Bulk updates with intuitive color-coded visualization.
    \item \textbf{Standing Area Modifications:} Easy editing of boundaries and capacities.
    \item \textbf{Seamless Integration:} Ensures consistent updates via \texttt{Map Context}.
\end{itemize}

With its structured approach and live updates, the detail editor ensures flexible and efficient stadium configuration.


\subsection{Toolbar}
\setauthor{Michael Ruep}
\label{sec:toolbar}
The \textbf{ToolBar.tsx} component provides an intuitive interface for tool selection, map management, and saving operations within SeatGen. It enables users to interact efficiently with the stadium layout by offering a structured toolbar for tool activation, map naming, and data persistence.

\textbf{Key Responsibilities:}
\begin{itemize}
    \item \textbf{Tool Selection:} Provides quick access to various editing tools.
    \item \textbf{Map Title Editing:} Allows renaming the map for better organization.
    \item \textbf{Saving and Previewing:} Ensures seamless data persistence and view-only mode.
\end{itemize}

\subsubsection{Component Structure}
The \textbf{ToolBar.tsx} consists of the following elements:
\begin{itemize}
    \item \textbf{Tool Icons:} Represent different editing modes (selection, adding seats, grid placement, area selection, standing area placement).
    \item \textbf{Map Name Editor:} Enables modifying the displayed name of the stadium layout.
    \item \textbf{Save and Preview Buttons:} Handles storing changes and switching to a preview mode.
\end{itemize}

\begin{figure}[H]
    \centering
    \includegraphics[width=0.7\textwidth]{pics/toolbar01.png}
    \caption{Toolbar Icons in ToolBar.tsx}
    \label{fig:toolbar-icons}
\end{figure}

Figure~\ref{fig:toolbar-icons} illustrates the available tool icons, where users can select different functionalities. The currently active tool is highlighted.

\subsubsection{Tool Selection and Activation}
The toolbar supports multiple tools, each represented by an icon. Users can switch between tools to perform various stadium layout modifications.

\begin{lstlisting}[language=TypeScript, caption=Defining Available Tools, label=lst:toolbar-tools]
const tools: Tool[] = [
    {
        id: "addTool",
        icon: <PlusIcon></PlusIcon>,
        onSelect: () => handleToolSelect((e) => {
            props.addSeat(e.latlng.lat, e.latlng.lng)
        }, "addTool", "cell"),
        hotkey: "c"
    },
    {
        id: "addGridTool",
        icon: <TableCellsIcon></TableCellsIcon>,
        onSelect: () => handleToolSelect(() => {
        }, "addGridTool", "cell"),
        hotkey: "g"
    },
    {
        id: "standingAreaTool",
        icon: <UserGroupIcon />,
        onSelect: () => handleToolSelect(() =>console.log("standingtool"), "standingAreaTool", "crosshair"),
        hotkey: "s"
    }
];
\end{lstlisting}

\textbf{How It Works:}
\begin{itemize}
    \item Each tool has an icon and an activation function.
    \item Hotkeys allow quick switching between tools.
    \item Selected tools modify the user’s interaction mode (e.g., clicking, selecting, adding seats).
\end{itemize}

\subsubsection{Editing the Map Name}
The toolbar includes an editable text field for renaming the stadium layout. Clicking the pencil icon toggles the edit mode.

\begin{lstlisting}[language=TypeScript, caption=Editing Map Name, label=lst:toolbar-edit-name]
{
    !editName ? (
        <>
            <p>{displayName}</p>
            <PencilIcon onClick={() => setEditName(true)} className="mapNameEdit cursor-pointer" />
        </>
    ) : (
        <>
            <input
                autoFocus
                type='text'
                className="text-center"
                value={tempDisplayName}
                onChange={(e) => setTempDisplayName(e.target.value)}
            />
            <CheckIcon onClick={() => {
                seatgenApiClient.api.updateMapName({ bucketName: props.mapDto.bucketName, mapName: props.mapDto.mapName, displayName: tempDisplayName })
                    .then((r) => setDisplayName(r.data.second))
                setEditName(false);
            }} className="cursor-pointer" />
        </>
    )
}
\end{lstlisting}

\textbf{How It Works:}
\begin{itemize}
    \item Clicking the pencil icon allows editing the map title.
    \item The name is updated and saved via an API request.
    \item The change is immediately reflected in the UI.
\end{itemize}

\subsubsection{Save and Preview Buttons}
The toolbar also contains buttons for saving progress and switching to a preview mode.

\begin{figure}[H]
    \centering
    \includegraphics[width=0.5\textwidth]{pics/toolbar02.png}
    \caption{Save and Preview Buttons in ToolBar.tsx}
    \label{fig:toolbar-save-preview}
\end{figure}

\begin{lstlisting}[language=TypeScript, caption=Saving and Previewing, label=lst:toolbar-save-preview]
<button className={SgButtonType.primarySmall} onClick={() => {
    seatgenApiClient.api.saveSeats(context.seats.map((s) => ({
        id: s.id,
        x: s.position.lat,
        y: s.position.lng,
        categoryId: s.category?.id
    })))
    .then(context.updateSaveIndex);
}} disabled={isSaveLoading}>
    {isSaveLoading ? <Spinner size={4} /> : <span>Save</span>}
</button>

<button className={SgButtonType.primarySmall} onClick={() => {
    props.setEditable(false);
    setSelectedTool(null);
}}>
    Preview
</button>
\end{lstlisting}

\textbf{How It Works:}
\begin{itemize}
    \item The \texttt{Save} button persists all seat and standing area modifications to the backend.
    \item A checkmark icon confirms successful saving, while a warning icon indicates unsaved changes.
    \item The \texttt{Preview} button switches the interface to a non-editable mode.
\end{itemize}

\subsubsection{Final Summary of ToolBar.tsx}
\textbf{ToolBar.tsx} plays a critical role in SeatGen by providing an accessible interface for stadium layout editing. It includes:
\begin{itemize}
    \item \textbf{Interactive Tool Selection:} Enables switching between different seat and area modification tools.
    \item \textbf{Map Naming:} Allows users to rename and organize stadium layouts.
    \item \textbf{Save and Preview Mechanism:} Ensures data integrity and view-only mode.
\end{itemize}

By offering an efficient and intuitive interface, \textbf{ToolBar.tsx} enhances the overall usability of SeatGen.

