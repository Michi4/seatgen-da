In the development of our application, we incorporated various design patterns to address specific challenges efficiently. Many sectors of the application required distinct patterns, including the interactive editor and tool functionalities. Because the interactive editor and tool system included a lot of complex features, we needed specific solutions. Undoing actions, for instance, is a widely expected usability feature in almost every modern editor, whether text-based or visual. This feature is applied broadly across software products, and various solutions exist. Special implementations of design patterns were utilized for these following parts of the application:
\begin{compactitem}
\item Undo/Redo functionality of Actions
\item Tool System
\item Backend services
\end{compactitem}

\subsection{Undo/Redo}

The undo/redo mechanism is essential for usability, providing users with the flexibility to revert and reapply actions efficiently. Multiple approaches exist for implementing this feature:

\begin{compactenum}
\item \textbf{State Snapshot Approach}: This method involves saving the entire application state at each change and reverting to the previous state when undoing. While simple to implement, this approach is inefficient due to excessive memory consumption and redundant data storage. An advantage is, that old states can easily be restored without any additional logic and calculations. This makes it not very prone to bugs and errors.
\item \textbf{Differential State Storage}: Instead of storing complete states, this approach records only the differences between successive states, similar to version control systems such as Git. While more efficient, this method becomes complex as the number and types of objects increase (in this case it would be Seats and Standing-Areas).
\item \textbf{Command Pattern}: Actions are encapsulated as objects that implement a common interface, containing methods for execution and reversal. This approach allows flexible and scalable undo/redo functionality, making it ideal for complex interactive applications. It also allows to execute additional business logic when undoing an action, like deleting additional data that was created by the action, or sending requests to a backend. This makes it an excellent choice when the states are distributed.
\item \textbf{Memento Pattern}: This pattern captures and externalizes an object's internal state so that it can be restored later without violating encapsulation. While useful for preserving an object's complete state, it can be memory-intensive when storing multiple versions.
\end{compactenum}

Given the application's complexity, a variation of the Command Pattern has been implemented for the undo/redo functionality. This approach ensures scalability, efficiency, and maintainability while minimizing redundancy of data and code.

Our implementation defines an abstract \texttt{Action} class that all user actions must implement. This class enforces the inclusion of \texttt{execute} and \texttt{undo} methods, ensuring a standardized approach to action management.

\begin{lstlisting}[language=TypeScript,caption={Action class},label={lst:action-class}]
export abstract class Action {
    execute: (() => void) | undefined;
    undo: (() => void) | undefined;
}
\end{lstlisting}

For both of these properties a function is expected that can be called to execute the action or to \texttt{undo} the \texttt{action}. This is flexible approach and can be used for a lot of different actions. 

Another importance for this application was to allow the execution of business logic while undoing specific actions, like sending requests to the backend. This is very easy to impliment because each Action has its individual \texttt{execute} and \texttt{undo} function.

While this seems like a lot more logic is needed than in the other design patterns, the logic demanded by this is actually important for usability reasons, and lots of it is reusable. Because the undo function should be able to be executed manually by the user, calling the \texttt{undo()} shouldn't be the only way to reverse an action, for example when creating a new seat, you should be able to delete it again by calling the \texttt{undo()} function as well as a separate way like a delete button. So the developer should always provide both ways for usability reasons. This approach incentivizes developers to do it because it's mandatory to implement the logic for undoing and action anyway.

Here is an implementation of an action that creates a Standing-Area with its counterpart action that deletes the standing area. Both actions can reference each other for the undoing part to reduce code redundancy, because the opposite of creating a standing area is deleting it. That's why deleting it is the undo action of creating it and the other way around.

\begin{lstlisting}[language=TypeScript,caption={Add standing-area action implementation},label={lst:add-standing-area-action-implementation}]
export class AddStandingAreaAction implements Action {
    private readonly _newArea: StandingArea;
    private readonly _context: MapContextValue;

    constructor(newArea: StandingArea, context: MapContextValue) {
        this._newArea = newArea;
        this._context = context;
    }

    execute = () => {
        this._context.setStandingAreas((prevAreas) => 
            [...prevAreas, this._newArea]
        );
        if (this._context.standingAreasToDelete.includes(this._newArea.id))
        {
            this._context.setStandingAreasToDelete((prev) =>
                prev.filter(id => id !== this._newArea.id)
            );
        } else {
            this._context.setStandingAreasToSave((prev) => {
                const updated = [...prev, this._newArea.id];
                return updated;
            });
        }
    };

    undo = () => {
        new DeleteStandingAreaAction([this._newArea], this._context).execute()
    };
}
\end{lstlisting}

\begin{lstlisting}[language=TypeScript,caption={Delete standing-area action implementation},label={lst:delete-standing-area-action-implementation}]
export class DeleteStandingAreaAction implements Action {
    private readonly _deletedAreas: StandingArea[];
    private _undoAreas: StandingArea[] | undefined;
    private readonly _context: MapContextValue;

    constructor(deletedAreas: StandingArea[], context: MapContextValue) {
        this._deletedAreas = deletedAreas;
        this._context = context;
    }

    execute = () => {
        const deletedAreaIds = this._deletedAreas.map(area => area.id);
        this._context.setStandingAreas((prevAreas) =>
            prevAreas.filter((area) => !deletedAreaIds.includes(area.id))
        );
        this._context.setStandingAreasToDelete((prev) => 
            [...prev, ...deletedAreaIds]
        );
        deletedAreaIds.forEach(id => {
            if (this._context.standingAreasToSave.includes(id)) 
            {
                this._context.setStandingAreasToSave(this._context
                .standingAreasToSave
                .filter(savedId => savedId !== id));
            } else 
            {
                this._context.setStandingAreasToDelete((prev) => [...prev, id]);
            }
        });
    }

    undo = () => {
        if (this._undoAreas) {
            this._deletedAreas.forEach((area) => {
                new AddStandingAreaAction(area, this._context).execute()
            })
        }
    }
}
\end{lstlisting}

Int the code in Listing \ref{lst:add-standing-area-action-implementation} and \ref{lst:delete-standing-area-action-implementation} you have the functions with the business logic for creating and deleting a standing area. When the \texttt{undo()} function is called, actually a new \texttt{DeleteStandingAreaAction} is created, and it's execute is called, because it implements the correct business logic for undoing the action. Same is true for the call of the call of the \texttt{undo()} function in the \texttt{DeleteStandingAreaAction}. With this code both functionalities can be implemented by separate buttons or something similar, and the undo and redo functionality is implemented as well. The needed contexts and functions to execute the business logic correctly for both classes can be defined individually in the constructor of the classes. The class also has to store the information to undo its actions, for example the move action has to store the old position of the object to be able to move it back to the old position.

The actual undoing logic is defined by a controller. When an action should be undo and redoable it has to be passed to the controller. The controller manages the function and can be called to undo or redo the last action. It also manages the stack of actions, so that all the actions can be undone and then redone again, until a new action is executed. When this happens the controller ignores all of the ``future'' actions that would have come after the current action. For example: Action1, Action2, Action3 and Action4 have been executed. The latest action saved by the Controller is currently Action4. Currently, all the actions can be undone and then redone in a stack like way. This means Action4 is undone, then Action3, then Action2 and so on. Then they can all be reapplied in the same revered order. When actions have been undone, to Action2 for example, and then a new Action5 is executed, Action3 and Action4 will be scrapped, because a new ``future'' has been created. This is a common behavior in undo and redo functionalities.

The implementation of the controller is as shown in listing \ref{lst:action-controller}.

\begin{lstlisting}[language=TypeScript,caption={Action controller implementation},label={lst:action-controller}]
const loopSize = 50;
const actionHistory: (Action | undefined)[] = new Array(loopSize).fill(undefined);

let historyIndex = 0;
let maxIndex = 0;
let minIndex = -1;
let savedIndex = 0

const loopSize = 50;
const actionHistory: (Action | undefined)[] = new Array(loopSize).fill(undefined);

let historyIndex = 0;
let maxIndex = 0;
let minIndex = -1;
let savedIndex = 0


const doAction = (action: Action) => {
    historyIndex = increase(historyIndex);

    while (maxIndex != historyIndex) {
        actionHistory[maxIndex] = undefined
        maxIndex = decrease(maxIndex)
    }

    actionHistory[minIndex] = undefined
    minIndex = increase(maxIndex);
    actionHistory[historyIndex] = action;
    action.execute!();
    checkForUnsavedChanges()
};

const updateSaveIndex = () => {
    savedIndex = historyIndex
    checkForUnsavedChanges()
}

function increase(num: number): number {
    return num !== loopSize - 1 ? num + 1 : 0;
}

function decrease(num: number): number {
    return num !== 0 ? num - 1 : loopSize - 1;
}

const undo = () => {
    if (historyIndex !== minIndex && actionHistory[historyIndex] !== undefined) {
        actionHistory[historyIndex]!.undo!()
        historyIndex = decrease(historyIndex);
        checkForUnsavedChanges()
        enqueueSnackbar("Undone", {variant: "info"})
    }
};

const redo = () => {
    if (historyIndex !== maxIndex && actionHistory[increase(historyIndex)] !== undefined) {
        historyIndex = increase(historyIndex);
        actionHistory[historyIndex]!.execute!();
        checkForUnsavedChanges()
    }
};

const getCurrentAction = (): Action | null => actionHistory[historyIndex] ?? null;
\end{lstlisting}

To register a new Action in the controller, the \texttt{doAction} function has to be called with the action as a parameter like in this listing \ref{lst:register-action}. The context referenced here is the context containing business logic for the map as well as containing the logic for the undo and redo controller.

\begin{lstlisting}[language=TypeScript,caption={Registering a new action in the controller},label={lst:register-action}]
context.doAction(new AddSeatAction(context.setSeats, lat, lng))
\end{lstlisting}

This controller stores all the actions in the form of a loop, with the size defined by the \texttt{loopSize} variable. The variable is set to 50 because more than 50 undoable actions back are not necessary. A circular buffer for storing this kind of data is very advantageous because when the buffer is full, the oldest action is overwritten by the newest action because the oldest data is not needed anymore. Other very important variables are the \texttt{minIndex} and \texttt{maxIndex} variables. They define the range of the actions that can be undone and redone. When undoing, it's checked that the current index which is represented by the \texttt{historyIndex} is not the same as the \texttt{minIndex} and there is also an undoable action, because then there would be no more actions to undo. Only if these conditions are fulfilled, there are actions to undo, and the \texttt{undo()} this is called, and the \texttt{historyIndex} is decreased. A similar logic is applied for the \texttt{redo()} function, but with the \texttt{maxIndex} and the \texttt{increase()} function. The \texttt{doAction()} function is used to handle new actions. It increases the \texttt{historyIndex} and sets the \texttt{maxIndex} to the \texttt{historyIndex} and overwriting all the no longer needed Actions with undefined. The \texttt{minIndex} is set to the \texttt{increase(maxIndex)} to ensure that when the loop is full, that the changes that are too old they are removed. At last the action is added to the list of actions and the \texttt{execute()} function is called.

The \texttt{increase()} and \texttt{decrease()} functions are used to increase and decrease the index in a circular way. This is necessary because the buffer is circular and when the end of the buffer is reached, the index has to be set to the beginning of the buffer again. This is done by checking if the index is at the end of the buffer and then setting it to the beginning of the buffer again.