

The \textbf{MapComponent.tsx} is one of the most critical components in SeatGen, responsible for rendering and managing the stadium seating map. It integrates with \textbf{Leaflet.js} to provide interactive seat visualization, selection, and manipulation.

\textbf{Key Responsibilities:}
\begin{itemize}
    \item Initializes and manages the \textbf{Leaflet map instance}.
    \item Loads seat and standing area data dynamically from the backend.
    \item Renders seats and standing areas as \textbf{interactive markers and polygons}.
    \item Handles user interactions such as \textbf{clicking, selecting, and dragging seats}.
    \item Implements \textbf{state synchronization} with the global context (\textbf{MapContext.tsx}).
    \item Provides support for \textbf{tool-based seat editing} (e.g., adding, deleting, moving).
\end{itemize}

\textbf{Component Structure:}
\begin{itemize}
    \item \textbf{State Management:} Uses \texttt{useState}, \texttt{useEffect}, and \texttt{useContext} to manage map data.
    \item \textbf{Leaflet Integration:} Utilizes \texttt{MapContainer} from \texttt{react-leaflet} for seamless map rendering.
    \item \textbf{Performance Enhancements:} Implements \texttt{useCallback} and \texttt{useRef} to optimize re-renders.
\end{itemize}

\subsubsection{Map Initialization and Leaflet Integration}

The \textbf{MapComponent.tsx} initializes the Leaflet map using the react-leaflet \texttt{MapContainer} component.

\begin{lstlisting}[language=TypeScript, caption=Initializing Leaflet Map in React, label=lst:react-leaflet]
const MapComponent: FC<MapProps> = ({ editable: initialEditable }) => {
    const context = useMapContext(); // Access global state (MapContext.tsx)
    const { bucketName, mapName } = useParams(); // Retrieve map identifiers from URL params

    const [editable, setEditable] = useState(initialEditable ?? true);
    const mapRef = useRef<L.Map>(null);

    return (
        <MapContainer
            crs={L.CRS.Simple} // Uses a flat, pixel-based coordinate system
            className="leaflet-container h-full w-full"
            ref={mapRef}
            center={[context.mapInfo.tileSize / (-2), context.mapInfo.tileSize / (2)]}
            zoom={context.mapInfo.defaultZoom}
            maxZoom={context.mapInfo.maxZoom}
            minZoom={context.mapInfo.minZoom}
            scrollWheelZoom={true}
            zoomControl={false}
            doubleClickZoom={false}
            preferCanvas={true}
            dragging={true}
            tap={false}
            renderer={L.canvas()} // Use Canvas for better performance
        >
            <TileLayer tms={true} url={`${context.mapInfo.mapDto.baseUrl}/{z}/{x}/{y}.png`} />
        </MapContainer>
    );
};
\end{lstlisting}

\textbf{Custom CRS (Coordinate Reference System):}
\begin{itemize}
    \item Uses \textbf{L.CRS.Simple}, a \textbf{2D pixel-based coordinate system}.
    \item Unlike traditional geographic maps, SeatGen does not need a curved map.
    \item All coordinates can be converted to a Cartesian X/Y grid.
\end{itemize}

\textbf{Custom Rendering Engine:}
\begin{itemize}
    \item Uses \textbf{L.canvas()} instead of SVG for performance.
    \item Enables handling of thousands of seat markers.
    \item Reduces DOM load by rendering elements in a drawing surface.
\end{itemize}

\textbf{Dynamically Loading Map Tiles:}
\begin{itemize}
    \item The tile URL is dynamically constructed based on \texttt{context.mapInfo}.
    \item Tiles are loaded asynchronously to improve map load times.
\end{itemize}


\subsubsection{Fetching Seat and Standing Area Data}
\setauthor{Michael Ruep}

The MapComponent.tsx fetches seat and standing area data from the backend when the component mounts. This is done using the api client in the useEffect hook.

\begin{lstlisting}[language=TypeScript, caption=Fetching Seat and Category Data, label=lst:fetch-seats]
useEffect(() => {
    if (bucketName && mapName) {
        // Fetch basic stadium map information
        seatgenApiClient.api.info(bucketName, mapName)
            .then(response => context.setMapInfo(response.data))
            .catch(error => console.error('Error fetching map info:', error));

        // Fetch seat categories before retrieving individual seats
        seatgenApiClient.api.getAllCategories().then((r) => {
            if (r.ok) {
                context.setCategories(r.data);
                seatgenApiClient.api.getSeatsByMap(bucketName, mapName).then(response => {
                    if (r.ok) {
                        context.setSeats(response.data.map((s) => ({
                            id: s.seatId!,
                            position: { lat: s.xcoord!, lng: s.ycoord! },
                            category: r.data.find(it => it.id === s.categoryId) ?? null
                        })));
                    }
                });
            }
        });

        // Fetch standing areas
        seatgenApiClient.api.getSectorByMap(bucketName, mapName).then((r) => {
            if (r.ok) {
                context.setStandingAreas(r.data.map((data) => ({
                    id: data.id!,
                    name: data.name!,
                    capacity: data.capacity,
                    coordinates: data.coordinates?.map((c) => new L.LatLng(c.x!, c.y!)) ?? [],
                    selected: false
                })));
            }
        });

    } else {
        console.error("BucketName or MapName not set");
    }
}, [bucketName, mapName]);
\end{lstlisting}

\textbf{Breakdown of Logic:}
\begin{itemize}
    \item Fetches stadium metadata (mapInfo) and sets it globally.
    \item Retrieves seat categories.
    \item Retrieves seat positions from the backend and maps them into React state.
    \item Ensures data consistency by linking seats to their corresponding categories.
\end{itemize}

\textbf{Example of Mapped Seat Object:}
\begin{lstlisting}[language=TypeScript, caption=Seat Object in State, label=lst:seat-object]
{
    id: 1234,
    position: { lat: 48.3069, lng: 14.2858 }, // Example coordinates
    category: { id: 2, name: "VIP", color: "#FFD700" } // Associated category
}
\end{lstlisting}

\subsubsection{Fetching Standing Area / Sector Data}
In addition to seats, the component also retrieves standing areas, which are handled separately.

\begin{lstlisting}[language=TypeScript, caption=Fetching Standing Areas, label=lst:fetch-standingareas]
seatgenApiClient.api.getSectorByMap(bucketName, mapName).then((r) => {
    if (r.ok) {
        context.setStandingAreas(r.data.map((data) => ({
            id: data.id!,
            name: data.name!,
            capacity: data.capacity,
            coordinates: data.coordinates?.map((c) => new L.LatLng(c.x!, c.y!)) ?? [],
            selected: false
        })));
    }
});
\end{lstlisting}

\textbf{Explanation:}
\begin{itemize}
    \item The API returns a list of sector polygons.
    \item Each area consists of a unique ID, name, capacity, and a list of coordinates.
    \item Coordinates are transformed into Leaflet’s LatLng format to be rendered as a polygon.
\end{itemize}

\textbf{Example of a Standing Area Object in State:}
\begin{lstlisting}[language=TypeScript, caption=Standing Area Object in State, label=lst:standingarea-object]
{
    id: 5678,
    name: "Sektor 1",
    capacity: 500,
    coordinates: [
        { lat: 48.3069, lng: 14.2858 },
        { lat: 48.3075, lng: 14.2862 },
        { lat: 48.3080, lng: 14.2856 }
    ],
    selected: false
}
\end{lstlisting}

\subsubsection{State Management and Performance}
\begin{itemize}
    \item \texttt{useEffect} Dependency Array:
    \begin{itemize}
        \item Ensures the API calls only run when \texttt{bucketName} or \texttt{mapName} change.
        \item Prevents unnecessary re-fetching on every render.
    \end{itemize}
    
    \item Efficient State Updates:
    \begin{itemize}
        \item Avoids unnecessary re-renders by batching state updates for seats and standing areas.
        \item No prop-drilling by storing globaly needed fetched data in the \texttt{MapContext}.
    \end{itemize}

    \item Error Handling:
    \begin{itemize}
        \item If any API call fails, an error is logged, and the operation is skipped.
        \item Ensures that failures in one request do not crash the entire component.
    \end{itemize}
\end{itemize}

\subsubsection{Rendering Seats and Handling Selection}
\setauthor{Michael Ruep}

Each seat in the stadium is rendered as a Leaflet marker, allowing users to interact with them dynamically. The selection mechanism is designed to provide an intuitive experience while supporting multi-selection for bulk operations.

\textbf{Rendering Seat Markers:}
\begin{lstlisting}[language=TypeScript, caption=Rendering and Selecting Seats, label=lst:seat-rendering]
const handleSeatClick = useCallback((id: number, event: L.LeafletMouseEvent) => {
    const isCtrlPressed = event.originalEvent?.ctrlKey;

    context.setSeats(prevSeats => prevSeats.map(seat => {
        if (seat.id === id) {
            const selected = isCtrlPressed ? !seat.selected : true;
            return { ...seat, selected };
        }
        return isCtrlPressed ? seat : { ...seat, selected: false };
    }));

    setOpenSideBar(true);
}, [context]);
\end{lstlisting}

\textbf{How This Works:}
\begin{itemize}
    \item Clicking a seat toggles its selected state.
    \item Holding \textbf{Ctrl} allows multi-selection, useful for bulk actions.
    \item Clicking a seat opens the \textbf{DetailEditor sidebar} for further modifications.
\end{itemize}
