To integrate Leaflet\cite{LeafletGitHub} in SeatGen, the React Leaflet library\cite{ReactLeafletDocs2} was utilized as a wrapper for Leaflet, due to an easier React implementation. The library provides a set of React components for Leaflet maps, instead of just having to use the JavaScript functions of the Leaflet library. To integrate a basic map, a \texttt{MapContainer} component and a map reference are required. The \texttt{MapContainer} is the area in the frontend where the map is displayed. The map reference is used to interact with the map, like adding layers or markers. The following code snippet shows how to create a basic map with the React Leaflet library. To make a map appear, there needs to be a \texttt{TileLayer} as a child of the \texttt{MapContainer}. Multiple \texttt{TileLayer} components can be used to display different map layers, but this is not required for the use case of SeatGen. The one required property of the \texttt{TileLayer} is the \texttt{url} property, which is the URL of the map tiles. In the \texttt{url} property, the \texttt{x}, \texttt{y} and \texttt{z} values are defined by the \texttt{\{x\}}, \texttt{\{y\}} and \texttt{\{z\}} placeholders. The \texttt{z} is the zoom level, \texttt{x} and \texttt{y} are the coordinates of the tile. A key part of the integration is the use of a map reference (mapRef), which allows programmatic interaction with the map instance. This reference is created using React’s useRef hook and is used to manipulate the map dynamically, such as adding layers, adjusting zoom levels, or panning to specific locations.

For the integration of custom markers in the map, the \texttt{Marker} component is used. The \texttt{Marker} component is a child of the \texttt{MapContainer} and has a position property.

Many different components utilize positions to display them on the map. Leaflet has two kinds of positions, the \texttt{LatLng} and the \texttt{Point}. The \texttt{LatLng} is a geographical point with a latitude and longitude. The \texttt{Point} is a point with x and y coordinates in pixels. The \texttt{Point} is used to position elements on the map, like markers or popups. 

Latitude and longitude are used for the representation of Earth's surface. Latitude specifies the north-south position and ranges from -90° (South Pole) to +90° (North Pole). Longitude specifies the east-west position and ranges from -180° to +180°. These coordinates are used in geographic coordinate systems, which are essential for positioning objects on a global scale, but not useful for the use case of SeatGen. If the coordinates are not transformed correctly, moving a marker in a straight line may result in a curved path. This is because of the aforementioned logic of the surface of the Earth. To avoid this, the \texttt{LatLng} coordinates can be converted to \texttt{Point} coordinates by providing the map reference. An example of such a conversion in the code of SeatGen is shown in Listing \ref{lst:latlng-point}.

\begin{lstlisting}[language=TypeScript, caption={Latitude Longitude and Point conversion}, label={lst:latlng-point}]
    //Point to LatLng
    const latLngPosition = map.layerPointToLatLng(new L.Point(x, y));

    //LatLng to Point
    const pointPosition: Point = map.latLngToLayerPoint(new L.Lat(lat, lng));
\end{lstlisting}

Leaflet also provides many features, some of which can be used in the editor without much reconfiguration for SeatGen's use case. Others must be reworked, or only partially used, with the remaining parts rewritten. Some of the features that could be used without modification include:
\begin{compactitem}
\item Zoom
\item Movement in the map
\item Tooltips of markers
\end{compactitem} 

SeatGen has a lot more of Leaflet's features implemented, but they are heavily modified. For example, the marker feature was utilized for displaying seats, but aside from the base features, everything else isn't provided by Leaflet and is implemented here instead.

\subsection{Writing Extensions}
\setauthor{Michael Stenz}
For the bigger changes inside Leaflet itself, SeatGen uses extensions to modify existing features or even overwrite them. Leaflet provides an easy way for developers to modify its functionalities. This can be done with the \texttt{extend} method that is provided by some Leaflet classes. When overwriting functions, knowledge of the internal Leaflet functionality is required. It's recommended to look into Leaflet's source code and study the class before overwriting it. When doing so, it is possible to create new subclasses of the existing class and integrate these new modified subclasses into the map. An example of this in SeatGen is in listing \ref{lst:leaflet-modification}.

\begin{lstlisting}[language=Typescript, caption={Modifying Leaflet Features},label={lst:leaflet-modification}]
L.Map.Multiselect = L.Map.BoxZoom.extend({

    _onMouseDown: function (e) {
        //...
        //Business logic for overwriting here
        //...
    },

    _onMouseMove: function (e) {
        //...
        //Business logic for overwriting here
        //...
    },

    _onMouseUp: function (e) {
        //...
        //Business logic for overwriting here
        //...
    },

    _finish: function () {
        //...
        //Business logic for overwriting here
        //...
    },

})

L.Map.mergeOptions({boxDemo: true});
L.Map.addInitHook('addHandler', 'boxDemo', L.Map.Multiselect);

L.Map.mergeOptions({boxZoom: false});

\end{lstlisting}


In the provided Leaflet extension code, mergeOptions is used to introduce and modify configuration options for the Leaflet Map class. This method allows developers to add custom options to existing Leaflet classes without modifying the core library. By merging options and using addInitHook, the new feature seamlessly integrates with Leaflet maps. The result is that whenever a map instance is created, the new feature replaces the default BoxZoom functionality if \texttt{boxDemo} is enabled.

\subsection{Event Handling}
\setauthor{Michael Stenz}
Leaflet provides an extensive event system that allows developers to listen for and handle various user interactions within the map. This event system is crucial for SeatGen, as it enables dynamic updates and interactions based on user actions. Events in Leaflet can be categorized into different types, such as mouse events, keyboard events, and map-specific events.

Commonly used events in SeatGen include:
\begin{compactitem}
\item \texttt{click}: Triggered when a user clicks on the map or an element.
\item \texttt{mousemove}: Fires whenever the mouse moves over the map, useful for hover effects.
\item \texttt{drag}: Used to detect when a user drags an element like a marker.
\end{compactitem}

Handling events in Leaflet is straightforward using \texttt{useMapEvents}. When using this hook and adding it to the Map, many events can be caught and handled. An example of using this hook with the \texttt{click} event is shown in listing \ref{lst:useMapEvents}.

\begin{lstlisting}[language=Typescript, caption={Handling Events in Leaflet},label={lst:useMapEvents}]
const MapEvents = () => {
    useMapEvents({
        click(e) {
            console.log("clicked")

            if (context.selectedToolId === "addTool" && mapClickCallBack) {
                console.log("Map clicked for add tool");
                mapClickCallBack(e);
            }

            if (context.selectedToolId === "default" || context.selectedToolId === "") {
                context.setSeats(prevSeats => prevSeats.map(seat => ({ ...seat, selected: false })));
                context.setSelectedStandingAreaIds([]);
            }

        },
    });
    return null;
};
\end{lstlisting}
