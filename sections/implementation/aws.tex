As already mentioned in the technology section, integrating AWS services into our project required a reliable way to interact with AWS APIs. The AWS SDK provides language-specific libraries that simplify communication with AWS services, including S3.

For administrators and developers, AWS provides multiple ways to interact with its services, each suited for different use cases:
\begin{compactitem}
\item \textbf{AWS Management Console (Web UI)} – A user-friendly graphical interface for managing AWS services, ideal for beginners or when making quick changes.
\item \textbf{AWS Command Line Interface (CLI)} – A powerful command-line tool that allows users to manage AWS resources via scripts and commands, enabling automation and repeatability.
\item \textbf{AWS SDKs – Language-specific libraries} (such as those for Python, Java, and JavaScript) that facilitate programmatic interaction with AWS services, making integration into applications seamless.
\item Others can be found in the AWS documentation.
\end{compactitem}

For configuring the S3 Pod, we used the AWS CLI tool. We chose this option over the AWS Management Console, because it is more efficient and already used queries can be saved in a text format. It's not as beginner-friendly as the Web UI, but it is way more powerful. When trying out to configure an S3 pod with the AWS CLI tool for the first time, it's recommended to read a lot of documentation to understand the concepts of AWS first, instead of just trying out commands, because there are lots of options that affect the security of the application. Worrying about cost was not necessary in this project, because the by the company provided I AM user didn't even have the rights to mess with the billing critical settings. Although a part of the cost is still dependent on our development process, but not the configuration of the bucket itself. 

To set up the CLI tool, the for the developers operating system appropriate installation method has to be chosen. The specifics are well documented on the AWS's documentation page. For the initial configuration of the CLI tool, the command \texttt{aws configure} has to be executed. This command will prompt the user to enter the access key, secret key, region, and output format. The access key and secret key can be obtained from the AWS Management Console. The region is the geographical location where the S3 bucket will be created, in this case it is \texttt{eu-central-1}, and the output format can be set to JSON, text, or table. The configuration is stored in two files located under linux in \texttt{\textasciitilde~/.aws/} directory. In this directory lie the \texttt{config} and \texttt{credentials} files. The \texttt{config} file contains the region and the output format, while the \texttt{credentials} file contains the access key and the secret key. The configuration can be changed at any time by executing the \texttt{aws configure} command again. These files can contain multiple profiles, for multiple developers. This is very useful when working in a team, or when working on multiple projects. The profile can be specified by adding the \texttt{-{}-profile} flag to the \texttt{aws} command. 

Other than for testing and managing bucket configuration, the AWS CLI was a very useful tool during development, because it allows the developer to manipulate the data in the buckets manually, as well as reading and listing the data with additional statistics.

Some of the useful utility commands used during the development process are listed in listing \ref{lst:aws-cli-commands}

\begin{lstlisting}[language=bash, caption={Usefull AWS CLI Commands}, label={lst:aws-cli-commands}]
# Lists all buckets
aws s3 ls --profile myprofile

# Command to recursively delete every item inside a directory
aws s3 rm --profile --recursive s3://ticketing-stadium-creator-dev/my-bucket/

# List all the data inside directory and provide statistics, like file size, object count and tota filesize
s3 ls --profile solvistas --summarize --human-readable --recursive s3://ticketing-stadium-creator-dev/my-bucket

# Count the number of objects in a bucket (similar but more compact results than in the previous command)
aws s3 ls --profile solvistas --recursive s3://ticketing-stadium-creator-dev/ | wc -l

# Listing all the applied bucket policies
aws s3api get-bucket-policy --profile solvistas --bucket ticketing-stadium-creator-dev

\end{lstlisting}

\subsection{S3 Bucket Configuration}
The configuration of an S3 bucket is a crucial step in the process of setting up an S3 pod. The bucket configuration determines the access permissions, storage classes, and other settings that affect how the bucket behaves. Here are some of the key configuration options that we used in our project:

To allow access to the S3 bucket for all users, we used the aws s3api put-bucket-policy command. This command applies a bucket policy that defines permissions for the bucket. In our case, we wanted to allow public read access to the objects in the bucket. The following command has been used \ref{lst:aws-cli-bucket-policy} with the \texttt{bucket-policy.json} configuration \ref{lst:bucket-policy}.

\begin{lstlisting}[language=bash, caption={AWS CLI command to set a bucket policy}, label={lst:aws-cli-bucket-policy}]
aws s3api put-bucket-policy --bucket ticketing-stadium-creator-dev --policy file://bucket-policy.json
\end{lstlisting}

\begin{lstlisting}[language=json,caption=Bucket policy JSON configuration, label=lst:bucket-policy]
{
    "Version": "2012-10-17",
    "Statement": [
        {
        "Effect": "Allow",
        "Principal": "*",
        "Action": "s3:GetObject",
        "Resource": "arn:aws:s3:::ticketing-stadium-creator-dev/*"
        }
    ]
}
\end{lstlisting}

This policy allows any user (\texttt{Principal: ``*"}) to perform the \texttt{s3:GetObject} action on all objects (\texttt{Resource: ``arn:aws:s3:::ticketing-stadium-creator-dev/*"}) within the bucket.

Amazon S3 provides \textbf{Access Control Lists (ACLs)} to manage access to buckets and objects. ACLs are a legacy access control mechanism, but they are still useful for simple use cases. Here are some important ACLs:

\begin{itemize}
    \item \textbf{Private}: The bucket and objects are accessible only by the bucket owner. This is the default ACL for new buckets.
    \item \textbf{Public Read}: The bucket and objects are readable by anyone on the internet. This is useful for hosting static websites or publicly accessible files.
    \item \textbf{Public Read-Write}: The bucket and objects are readable and writable by anyone on the internet. This is generally not recommended due to security risks.
    \item \textbf{Authenticated Read}: The bucket and objects are readable by any authenticated AWS user (not just the bucket owner).
\end{itemize}

While ACLs are easy to use, they are less flexible than bucket policies or IAM policies. For more granular control, it is recommended to use bucket policies or IAM policies. IAM (Identity and Access Management) policies provide fine-grained access control to AWS resources. Unlike bucket policies, which are attached to the bucket, IAM policies are attached to IAM users, groups, or roles. For this use case bucket policies were sufficient, because the bucket was only used to store static files, which should be accessible by everyone.