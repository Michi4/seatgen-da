The standing area tool enables users to define polygonal sectors directly on the map by selecting a series of points. These sectors, referred to as standing areas, are used to represent sections of a stadium that are not associated with fixed seating.

\textbf{Key Responsibilities:}
\begin{itemize}
    \item Allows users to draw a polygon on the map by clicking multiple points.
    \item Once finalized, dispatches an \texttt{AddStandingAreaAction} to insert the area into global state.
    \item Automatically assigns an id, empty name, and zero capacity by default.
    \item Supports full undo and redo through the action system.
\end{itemize}

\subsubsection{Capturing the Area}
Users activate the tool and click on the map to add polygon points:

\begin{lstlisting}[language=TypeScript, caption=Adding Polygon Points, label=lst:standingarea-add-point]
const handleMapClick = (event: L.LeafletMouseEvent) => {
    setPolygon([...polygon, event.latlng]);
};
\end{lstlisting}

Each click appends a new vertex to the polygon. The current polygon is stored in local state and displayed as an in-progress shape.

\subsubsection{Finalizing the Standing Area}
Once the polygon is complete, the user confirms by clicking a button that triggers the following logic:

\begin{lstlisting}[language=TypeScript, caption=Finalizing Standing Area, label=lst:standingarea-finalize]
const handleDone = () => {
    if (polygon.length < 3) return;

    const newArea: StandingArea = {
        id: Date.now(),
        name: "",
        capacity: 0,
        coordinates: polygon,
        selected: false
    };

    context.doAction(new AddStandingAreaAction(newArea, context.setStandingAreas));
    setPolygon([]);
};
\end{lstlisting}

\textbf{Explanation:}
\begin{itemize}
    \item Ensures at least three points are present to form a valid polygon.
    \item Constructs a new \texttt{StandingArea} object with default values.
    \item Dispatches an \texttt{AddStandingAreaAction} to insert the new area into state.
    \item Resets the drawing state for the next operation.
\end{itemize}

\subsubsection{Integration with the Action System}
The tool leverages the action-based architecture of SeatGen to ensure consistency and undoability. The actions are closer described in section \ref{sec:design-patterns}.

\begin{lstlisting}[language=TypeScript, caption=AddStandingAreaAction, label=lst:add-standingarea-action]
export default class AddStandingAreaAction extends Action {
    constructor(private standingArea: StandingArea, private setStandingAreas: React.Dispatch<React.SetStateAction<StandingArea[]>>) {
        super();
    }

    execute(): void {
        this.setStandingAreas(prev => [...prev, this.standingArea]);
    }

    undo(): void {
        this.setStandingAreas(prev => prev.filter(a => a.id !== this.standingArea.id));
    }
}
\end{lstlisting}

TODO: Satz

\textbf{Highlights:}
\begin{itemize}
    \item \texttt{execute()} appends the new area to the list.
    \item \texttt{undo()} removes the inserted area by filtering it out by id.
\end{itemize}

\subsubsection{Summary}
The standing area tool provides an intuitive polygon-drawing interface for defining sectors that are not associated with individual seats. Through real-time polygon tracking and integration with the action system, it supports flexible and reversible sector creation tailored to the needs of modern stadium planning.
