\section{Overview}
\setauthor{Michael Ruep}
Stadiums are typically operated by sports clubs, concert organizers, or third-party management companies. These actors want to sell their tickets efficiently and accurately for a large amout of offers. The environment is therefore characterized by:
\begin{itemize}
    \item \textbf{Varied Layouts:} Modern stadiums feature a curved layout, irregular seat patterns, and different pricing tiers. Managing the seat data in plain text is error-prone and time-intensive.
    \item \textbf{Frequent Configuration Changes:} Depending on ongoing events and seasons, stadium layouts are frequently restructured, requiring updated seat plans.
    \item \textbf{Limited Technical Staff:} The event organizers often rely on external developers to alter seat plan definitions, creating additional costs.
\end{itemize}

From a user-experience standpoint, these challenges make it clear that an intuitive, graphical editing interface is needed to replace the text-based seat data manipulation. This concept aligns with the principles of \emph{direct manipulation interfaces}, as described by Hutchins, Hollan, and Norman, which emphasize reducing the cognitive distance between user intent and system actions by allowing direct interaction with visual representations~\cite{Hutchins01121985}. In the context of SeatGen, users can place, move, and edit seats through an intuitive graphical interface rather than modifying abstract raw text data. Similar to how direct manipulation in statistical tools allows users to interact with data graphs instead of numbers, SeatGen provides a \textbf{spatially direct approach} to stadium seat planning.