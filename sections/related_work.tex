\section{Overview}
\setauthor{Michael Ruep}
Stadiums are typically operated by sports clubs, concert organizers, or third-party management companies. These actors want to sell their tickets efficiently and accurately for a large amount of offers. The environment is therefore characterized by:
\begin{itemize}
    \item \textbf{Varied Layouts:} Modern stadiums feature a curved layout, irregular seat patterns, and different pricing tiers. Managing the seat data in plain text is error-prone and time-intensive.
    \item \textbf{Frequent Configuration Changes:} Stadium layouts frequently change based on events and seasons, requiring continuous updates.
    \item \textbf{Limited Technical Staff:} The event organizers often rely on external developers to alter seat plan definitions, creating additional costs.
\end{itemize}

From a user-experience standpoint, these challenges make it clear that an intuitive, graphical editing interface is required to replace the text-based seat data manipulation. This concept aligns with the principles of \emph{direct manipulation interfaces}, as described by Hutchins, Hollan, and Norman, which emphasize reducing the cognitive distance between user intent and system actions by allowing direct interaction with visual representations \cite{Hutchins01121985}. In the context of SeatGen, users can place, move, and edit seats through an intuitive graphical interface rather than modifying abstract raw text data. Similar to how direct manipulation in statistical tools allows users to interact with data graphs instead of numbers, SeatGen provides a \textbf{spatially direct approach} to stadium seat planning.

\section{Stakeholder Analysis}
\setauthor{Michael Ruep}
Several parties interact with the SeatGen tool:
\begin{itemize}
    \item \textbf{Developers:} Historically, Solvistas' developers modified the seat layout text files. The new approach aims to minimize their involvement in the long term, except for initial and advanced configurations.
    \item \textbf{Event Organizers and Stadium Managers:} These stakeholders need the intuitive tools to make update to the seat maps without dealing with complex raw data formats.
    \item \textbf{Ticketing Platform Users:} The final seat layouts are used in Solvistas' ticketing service. Although these end-users do not edit the data themselves, the accuracy and clarity of seat layouts crucially impact their experience at the stadium.
\end{itemize}

By identifying these stakeholders and their needs, the main focus is placed on a graphical, user-friendly solution for seat map creation and maintenance.

\section{Technical Environment}
\setauthor{Michael Ruep}
On the technical side, the SeatGen application integrates with:
\begin{itemize}
    \item \textbf{AWS S3 Buckets:} Image tiles and map data are uploaded to and fetched from a secure Amazon Web Services (AWS) cloud environment.
    \item \textbf{React-based Frontend:} Provides an interactive user interface for managing and modifying seat layouts with real-time updates.  
    \item \textbf{Spring Boot / Kotlin Backend:} Manages image processing and seat data handling.  
    \item \textbf{PostgreSQL Database:} Stores seat configurations, categories, groups, sectors, and map data.  
\end{itemize}

In practice, large or complex venues may have thousands of individual seat entities, potentially impacting performance if the data management is not optimized. Additionally, the zoom levels demand loads of memory- and compute-efficient image slicing and compression, to avoid excessive storage usage on Amazon Web Services. The design choices reflect these constraints and requirements in multiple areas.

\section{Requirements and Challenges}
\setauthor{Michael Ruep}
A core requirement of SeatGen is “direct manipulation” \cite{Hutchins01121985}, which states that users interact more effectively when objects can be selected, dragged, and dropped in a way that mimics the real-world arrangement of physical seats or standing areas. Therefore the usability and user experience is a key Challenge. The following elements are particularly relevant:

\begin{itemize}
    \item \textbf{Immediate Feedback:} When a user modifies a seat position, the update is reflected instantly on the map. Dragging seats should feel instant and follow the user, mimicking real-life interactions
    \item \textbf{Simplicity and Learnability:} Familiar mouse-based interactions should allow users to perform specific and complex tasks—such as grouping seats, creating standing areas, or categorizing—without requiring specialized training or an understanding of coordinate systems.
    \item \textbf{Cognitive Offloading:} By representing seats visually, the mental effort to interpret raw text-based seat coordinates or stadium layouts is reduced significantly. Which improves efficiency and accuracy by a huge margin.
\end{itemize}

Leaflet, used within React, handles dynamic rendering and live interaction for the stadium seat map. Additionally, quick actions in the form of hotkeys further improve efficiency.

\section{Summary}
\setauthor{Michael Ruep}
In summary, the environment for SeatGen includes a mix of business and technical constraints, demanding a straightforward, user-friendly and yet powerful visual editor. In the following chapters, the technical and logical aspects of how the set requirements are met will be further described.