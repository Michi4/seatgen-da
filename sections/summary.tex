\section{Michael Ruep}
\section{Michael Stenz}
For my part I was responsible for most of the technology choices in the backend and some design choices in the frontend. All these choices had a huge impact on the entire development of the application because some featured centered around these libraries and patterns. In my opinion some choices were very good, but some choices probably weren't the most optimal as stated in the thesis, like java vs python for the backend. A big limitation were the technology demands of the Solvistas company, but it's important to also address these challenges and to find solutions that fit the company's needs. I personally learned a lot of new patterns that are used in the industry and I think that this new knowledge can be very useful in future projects. When it comes to splitting the tasks, it was a very smooth process because my partner and I had a very good communication and were able to both work on the tasks and tools that fitted more our expertise and interests. Also, after the minimal implementation of the project had been done, we were able to expand it feature by feature very easily because in the end the project more or less looked like a small framework, with all the patterns that enabled this easy expandability. A big takeaway from this project is that it can be overwhelming at first when working with huge libraries like Leaflet. When dealing with these challenges it's very important to start with a small prototype and read the documentation very carefully because most of the time the solution is hidden somewhere in the documentation, and a lot of time can be saved when reading it instead of trying to figure out the solution by yourself.

The decision to collaborate with a company was both good and bad in different aspects. On the one hand, it was very good because we were able to get a lot of feedback from the company, and we were able to address a real-world problem. On the other hand a part of our task was to understand parts of the Ticketing project which has a huge codebase, and we also had to understand code of the csv-stadium-creator tool and the workflow of the employees that are responsible for the maintenance and creation of the stadium.