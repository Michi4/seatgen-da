\section{Michael Ruep}

My focus in this project was on the frontend implementation of SeatGen. I was responsible for building the interactive stadium editing experience using React, TypeScript, and Leaflet. This included implementing the core component structure around the \texttt{MapComponent.tsx}, managing state updates via \texttt{MapContext.tsx}, and also partially integrating undo/redo functionality through a command pattern using actions.

One of the most challenging but rewarding parts was implementing the live multi-seat dragging logic. This required a deep understanding of Leaflet’s coordinate system and careful handling of relative positioning during real-time updates. Additionally, I worked on the \texttt{DetailEditor.tsx}, which provides a comprehensive interface for editing seats, assigning categories, managing groups, and configuring standing areas. I also did the \texttt{ToolBar.tsx} with my team partner, which ties all interactive tools together, and implemented the logic for triggering actions through tool selection.

Throughout the project, I learned how to structure complex, interactive applications in a scalable and modular way. I also improved my ability to collaborate on a shared codebase by ensuring that my parts were reusable and extensible. In terms of workflow, the communication and division of work between my colleague and me was seamless. We were able to work independently on different parts of the project at all time and merge our work together without any major conflicts.

Looking back, one of the most important things I learned is how important it is to fully understand a library like Leaflet before diving into implementation. There were many instances where understanding a subtle detail in the documentation helped save hours of programming. Overall, I’m very happy with the functionality I was able to implement and proud of how our diploma thesis turned out.
