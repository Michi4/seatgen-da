\section{Michael Ruep}

My focus in this project was on the frontend implementation of SeatGen. I was responsible for building the interactive stadium editing experience using React, TypeScript, and Leaflet. This included implementing the core component structure around the \texttt{MapComponent.tsx}, managing state updates via \texttt{MapContext.tsx}, and also partially integrating undo/redo functionality through a command pattern using actions.

One of the most challenging but rewarding parts was implementing the live multi-seat dragging logic. This required a deep understanding of Leaflet’s coordinate system and careful handling of relative positioning during real-time updates. Additionally, I worked on the \texttt{DetailEditor.tsx}, which provides a comprehensive interface for editing seats, assigning categories, managing groups, and configuring standing areas. I also did the \texttt{ToolBar.tsx} with my team partner, which ties all interactive tools together, and implemented the logic for triggering actions through tool selection.

Throughout the project, I learned how to structure complex, interactive applications in a scalable and modular way. I also improved my ability to collaborate on a shared codebase by ensuring that my parts were reusable and extensible. In terms of workflow, the communication and division of work between my colleague and me was seamless. We were able to work independently on different parts of the project at all times and merge our work together without any major conflicts.

Looking back, one of the most important things I learned is how important it is to fully understand a library like Leaflet before diving into implementation. There were many instances where understanding a subtle detail in the documentation helped save hours of programming. Overall, I’m very happy with the functionality I was able to implement and proud of how our diploma thesis turned out.

\section{Michael Stenz}
For my part I was responsible for most of the technology choices in the backend and some design choices in the frontend. All these choices had a huge impact on the entire development of the application because some featured centered around these libraries and patterns. In my opinion some choices were very good, but some choices probably weren't the most optimal as stated in the thesis, like java vs python for the backend. A big limitation were the technology demands of the Solvistas company, but it's important to also address these challenges and to find solutions that fit the company's needs. I personally learned a lot of new patterns that are used in the industry and I think that this new knowledge can be very useful in future projects. When it comes to splitting the tasks, it was a very smooth process because my partner and I had a very good communication and were able to both work on the tasks and tools that fitted more our expertise and interests. Also, after the minimal implementation of the project had been done, we were able to expand it feature by feature very easily because in the end the project more or less looked like a small framework, with all the patterns that enabled this easy expandability. A big takeaway from this project is that it can be overwhelming at first when working with huge libraries like Leaflet. When dealing with these challenges it's very important to start with a small prototype and read the documentation very carefully because most of the time the solution is hidden somewhere in the documentation, and a lot of time can be saved when reading it instead of trying to figure out the solution by yourself.

The decision to collaborate with a company was both good and bad in different aspects. On the one hand, it was very good because we were able to get a lot of feedback from the company, and we were able to address a real-world problem. On the other hand a part of our task was to understand parts of the Ticketing project which has a huge codebase, and we also had to understand code of the csv-stadium-creator tool and the workflow of the employees that are responsible for the maintenance and creation of the stadium.